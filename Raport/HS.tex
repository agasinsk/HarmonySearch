% !TeX encoding = UTF-8
% !TeX spellcheck = pl_PL

% $Id:$
%% Konfiguracja:
\newcommand{\formakursu}{Projekt}
\newcommand{\kurs}{Teoria i metody optymalizacji}
\newcommand{\doctype}{Harmony Search}
\newcommand{\osobaA}{Artur \textsc{Gasi\'nski}, 218685}
\newcommand{\osobaB}{Bartosz \textsc{Lenartowicz}, 218518}
\newcommand{\termin}{wt 9:15-11:00}
%wpisz imię i nazwisko prowadzącego
\newcommand{\prowadzacy}{Dr in\.{z}. Ewa \textsc{\'Szlachcic}}
\documentclass[10pt, a4paper]{article}
\usepackage{listings}

\include{preambula}
\begin{document}
\def\tablename{Tabela}	%zmienienie nazwy tabel z Tablica na Tabela
\begin{titlepage}
	\begin{center}
		\textsc{\LARGE \formakursu}\\[1cm]		
		\textsc{\Large \kurs}\\[0.5cm]		
		\rule{\textwidth}{0.08cm}\\[1cm]
		{\huge \bfseries \doctype}\\[1cm]
		\rule{\textwidth}{0.08cm}\\[1cm]
		\begin{flushright} \large
		\emph{Autor: }\\
		\osobaA\\
		\osobaB\\[0.4cm]
		\emph{Termin: }\termin\\[0.4cm]
		\emph{Prowadzący:} \\
		\prowadzacy \\
		\end{flushright}
		\vfill
		{\large \today}
	\end{center}	
\end{titlepage}
\newpage
\tableofcontents
\newpage

\section{Wstęp}
\label{sec:wstep}
Większość metod optymalizacji procesów produkcyjnych czy logistycznych sprowadza się do przedstawienia zależności pomiędzy procesami, którymi firma jest w stanie manipulować, za pomocą zmiennych. Następnie tworzone jest równanie matematyczne opisujące system firmy. Takie równanie nosi nazwę funkcji celu. By właściwie zoptymalizować proces należy znaleźć, w zależności co poddane jest optymalizacji, minimum bądź maksimum globalne takiej funkcji. Obliczanie minimalnej bądź maksymalnej wartości funkcji, zależnej od wielu zmiennych w pożądanym czasie, nie jest zadaniem łatwym. Istnieją specjalne algorytmy do rozwiązywania zadań tego typu. Niniejsza praca została napisana by przedstawić jeden z takich algorytmów -- Harmony Search. Wpierw w rozdziale \ref{sec:opis} został przedstawiony opis tego algorytmu. Rozdział \ref{sec:implementacja} przedstawia implementacje Harmony Search w programie komputerowym. (...)

\begin{figure}[htbp]
	\centering
	%\includegraphics[width=0.80\textwidth]{images/nazwa.png}
	%\caption{Podpis}
	\label{fig:nazwa}
\end{figure}

\section{Opis działania algorytmu}
\label{sec:opis}
Algorytm Harmony Search został przedstawiony w 2001 przez Zong Woo Geem, Joong Hoon Kim oraz G.V. Loganathan w pracy "A New Heuristic Optimization Algorithm: Harmony Search" \cite{bib:orginal}. Inspiracją dla algorytmu było szukanie przez muzyków jazzowych podczas improwizacji najlepszych harmonii dźwięków. Umożliwia on znajdowanie minimów lokalnych funkcji wielu zmiennych. Główna zasada polega wyszukiwaniu rozwiązań na podstawie wartości wcześniej obliczonych oraz modyfikacji nowych zmiennych z określonym z góry prawdopodobieństwem. Szczegółowe przedstawianie algorytmu mija się z celem ponieważ przebieg algorytmu został przetłumaczony na język polski np. w pracy \cite{bib:tlumaczenie}. W następnym rozdziale \ref{sec:implementacja} zostanie przedstawiona szczegółowo implementacja algorytmu. 

\section{Implementacja}
\label{sec:implementacja}
Program do obliczania minimum funkcji na podstawie algorytmu Harmony Search został napisany w języku Java. Technologią dostarczającą GUI dla użytkownika była {\em JavaFx}. Program korzystał również z zewnętrznej biblioteki do obliczania wartości funkcji w punkcie \cite{bib:mathparser}. Wykres prezentujący najlepsze rozwiązanie zaznaczone na wykroju z przestrzeni rozwiązań funkcji został stworzony za pomocą biblioteki {\em jzy3D} \cite{bib:jzy3d}.
\subsection{GUI}
\label{subsub:gui}

\begin{figure}[htbp]
	\centering
	\includegraphics[width=0.80\textwidth]{images/nazwa.png}
	\caption{Podpis}
	\label{fig:nazwa}
\end{figure}


\section{Przykładowe rozwiązania}
\label{sec:przyklady}

\section{Problemy podczas pracy}
\label{sec:problemy}

\section{Podsumowanie}
\label{sec:podsumowanie}

\newpage

\addcontentsline{toc}{section}{Bibilografia}
\bibliography{bibliografia}
\bibliographystyle{plain}

\end{document}

